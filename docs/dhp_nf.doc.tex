%% ================================================================================
%% This LaTeX file was created by AbiWord.                                         
%% AbiWord is a free, Open Source word processor.                                  
%% More information about AbiWord is available at http://www.abisource.com/        
%% ================================================================================

\documentclass[a4paper,portrait,12pt]{article}
\usepackage[latin1]{inputenc}
\usepackage{calc}
\usepackage{setspace}
\usepackage{fixltx2e}
\usepackage{graphicx}
\usepackage{multicol}
\usepackage[normalem]{ulem}
%% Please revise the following command, if your babel
%% package does not support en-GB
\usepackage[en]{babel}
\usepackage{color}
\usepackage{hyperref}
 
\begin{document}

\setlength{\oddsidemargin}{3.17cm-1in}
\setlength{\textwidth}{\paperwidth - 3.17cm-3.17cm}

On hearing true teachings,


the hearts of those who are receptive


become serene, like a lake,


deep, clear and still.





Dhammapada verse 82


\_\_\_\_\_\_\_\_\_\_\_\_\_\_\_\_\_\_\_\_\_\_\_\_\_\_\_\_\_\_\_\_\_\_\_\_\_\_\_\_\_\_\_\_\_\_\_\_\_\_\_\_\_\_\_\_\_\_\_\_\_\_





In memory of Prof. D. E. Gunatilleke


\\May he attain Nibbana


\\Printed for free distribution 


by his wife and children











Dhammapada Reflections





52 verses from the Dhammapada


with comments by





Ajahn Munindo

















Buddho logo








A R U N A   P U B L I C A T I O N S





\_\_\_\_\_\_\_\_\_\_\_\_\_\_\_\_\_\_\_\_\_\_\_\_\_\_\_\_\_\_\_\_\_\_\_\_\_\_\_\_\_\_\_\_\_\_\_\_\_\_\_\_\_\_\_\_\_\_\_\_\_\_\_\_\_\_\_





FOR FREE DISTRIBUTION


This publication is made available


for free distribution


by Aruna Publications,


administered by


Harnham Buddhist Monastery Trust.


Please see note at back for copyright details 


And  information about sponsorship of further books.





Cover photo offered by Andrew Binkley


www.andrewbinkley.com

















Sabbadhanam dhammadanam jinati


`The gift of Dhamma surpasses all other gifts.'


© Aruna Publications 2009











\_\_\_\_\_\_\_\_\_\_\_\_\_\_\_\_\_\_\_\_\_\_\_\_\_\_\_\_\_\_\_\_\_\_\_\_\_\_\_\_\_\_\_\_\_\_\_\_\_\_\_\_\_\_\_\_\_\_\_\_\_\_\_\_\_\_\_





Preface





It has long been the tradition in Buddhist countries for the laity to come to the local monastery each new-moon and full-moon to hear a Dhamma talk. Indeed the Buddha himself encouraged his Sangha to maintain this fortnightly practice. When it was suggested to me that I could use the internet to send out tracts of Dhamma each new-moon and full-moon, I wasn't sure about the idea, however I decided to give it a go. Although we live in a world where the phases of the moon no longer mean much, for many it still helps to be reminded of the ancient tradition of which we are a part. 


In September 2007 we started to send out verses from the Dhammapada selected from `A Dhammapada for Contemplation', 2006. Each `moon day' one verse was offered, supported by a short reflection on the verse. This programme is now quite well known, passed on from one person to the next by word of mouth and Cc-ed emails. I hear from people in different parts of the world who appreciate receiving a timely reminder of the ancient way as they go about their busy lives. Others look forward each new and full-moon evening to opening their emails when they get home from work. They are used privately, copied extensively, translated, and passed around. I hear also they form the basis for discussion at some weekly meditation group meetings.


	It has been my intention that by sharing my personal reflections in this way, others might feel encouraged to engage their own contemplative ability. There is a tendency perhaps for Buddhist practitioners in the West to try to find peace and understanding by stopping all thinking. Yet the Buddha tells us that it is by yoniso manasikara, or wise reflection, that we come to see the true nature of our minds; not through just stopping thinking.


	I am indebted to many who have helped in the preparation of this material. For the Dhammapada verses themselves I consulted several authoritative versions. In particular I have used the work by Ven. Narada Thera (B.M.S. 1978), Ven. Ananda Maitreya Thera (Lotsawa 1988), Daw Mya Tin and the editors of the Burmese Pitaka Association (1987) and Ajahn Thanissaro. For the recorded stories associated with the verses I also turned to \href{http://www.tipitaka.net/}{www.tipitaka.net}. 


When I had heard from enough people that a book version of these reflections would be useful it was to my good friend Ron Lumsden I turned. His considerable editing skill has helped craft my work making it ready for a more wide-spread audience.


	May the blessings that arise from the compilation of this small book be shared with all who have been involved in its production and sponsorship.  May all who seek the way find it and experience the freedom at its end. May all beings seek the way.





Bhikkhu Munindo


Aruna Ratanagiri Buddhist Monastery


Northumberland, UK


Rainy Season Retreat (vassa) 2009





\_\_\_\_\_\_\_\_\_\_\_\_\_\_\_\_\_\_\_\_\_\_\_\_\_\_\_\_\_\_\_\_\_\_\_\_\_\_\_\_\_\_\_\_\_\_\_\_\_\_\_\_\_\_\_\_\_\_\_\_\_\_\_\_\_\_\_





Introduction








The radiance of the Buddha's wisdom reaches us here today, lighting the path that guides us towards freedom from suffering. The luminosity of the Dhamma is reflected in the words and deeds of those who follow the Buddha's Teachings. Millions of men and women have caught a glimpse of this light in the midst of our dark world. Guided by the Teacher's words they have come to recognise the warm-heartedness and clarity that is their true nature.


	This book contains a selection of 52 verses taken from The Dhammapada. Accompanying each verse is a short paragraph by Ajahn Munindo, a Theravadin Buddhist monk and currently abbot of Aruna Ratanagiri monastery in Northumberland, UK. The complete Dhammapada contains 423 verses, each a timeless example of the Buddha's radiant wisdom. They are: 'very old artefacts that miraculously contain within them the profound utterances of the Buddha', as Thomas Jones says in the Afterword of ``A Dhammapada for Contemplation'' 2006, the publication from which these verses are taken. 


	The message of both the original verses and the comments Ajahn Munindo offers are received in a form that is neither preaching nor doctrinal. The verses are translations from an oriental language that is by its nature, culturally indirect. The format of the verses is curious and interesting, having the quality of the Japanese haiku or stanzas from the I Ching. They are inductive; they `induce' understanding in the reader. At times the sequence of the lines in a verse appears inverted. There's a sense that their beginnings should be at the end because the examples come before the underlying thought of the verse is stated; as we can see in verse 377: ``As old flowers fall from a jasmine plant, let lust and hatred fall away.'' The inductive style is softer than English readers are used to; here we are more gently led towards the main thought of the verse: ``... let lust and hatred fall away.'' Western languages in general use the deductive style, where the reader is more firmly directed to understanding. We can get an idea of this in verse 377, if we take the theme of the verse and place it at the beginning: ``Let lust and hatred fall away like old flowers falling from a jasmine plant.'' 


	The Dhammapada verses are aesthetic and indirectly instructive: ``Let go of that which is in front, let go of that which has already gone, and let go of in-between. With a heart that takes hold nowhere, you arrive at the place beyond all suffering.'' v. 348. The point of arrival (the place beyond all suffering) comes at the end of the verse. This inductive structure naturally enables the reader to reflect on `the arrival', having let go of everything.


	Ajahn Munindo's comments are presented in this same inductive form; they invite participation. Spiritual instructors of all persuasions can tend to talk too much about the possibility of the journey rather than joining us as we travel on the way. Readers of this book have an opportunity to see how being part of a shared enquiry can naturally give rise to new understanding; the process introduces us to a new part of ourselves. The Buddha made it very clear that it was up to us to make the effort: ``I can but point the way''. Dhammapada Reflections helps us to make this effort. In his preface Ajahn Munindo identifies wise reflection (yoniso manasikara) as a primary element of the spiritual journey. His comments plant the seeds for reflection. We are left to nurture and see them grow. 


	Perhaps you will decide to leave this volume open on your shrine table with a new verse to ponder on each week or maybe carry it with you as you travel. As one who has been studying these verses for many months now, I am sure you will come to find they are inspirational. They are beautiful things; their gently instructional nature simply and skillfully directs us. Step by step we approach Dhamma: that which has the power to enhance all aspects of our lives. 


On behalf of all the recipients of the fortnightly Dhammasakaccha messages, I would like to express our gratitude to Ajahn Munindo for the time and energy he has put into the preparation of the material for this publication. That which was previously available by email to a few hundred individuals can now be read and passed around in the form of this beautiful book.





Ron Lumsden


Little Oakley


Essex 


2009


\_\_\_\_\_\_\_\_\_\_\_\_\_\_\_\_\_\_\_\_\_\_\_\_\_\_\_\_\_\_\_\_\_\_\_\_\_\_\_\_\_\_\_\_\_\_\_\_\_\_\_\_\_\_\_\_\_\_\_\_\_\_\_\_\_\_\_








With an image of liberation as the goal 


the wise abandon darkness and cherish light,


leave petty security behind and seek freedom from attachment.


To pursue such release is difficult and rare, yet the wise will seek it, 


detaching themselves from obstructions, purifying heart and mind.





v. 87-88








The Buddha offers images that illustrate the goal, uplifting and supporting us in our effort to let go of that which obstructs and limits us. If we hold too tightly to the images, we may lose perspective on the here and now element of the journey; instead of actually doing the practice, we are imagining it. If we fail to give right emphasis to the goal, we may become lost in the distraction of sense objects -- agreeable and disagreeable. The pursuit of true freedom is difficult but consider how much suffering comes if we don't practice. With wise reflection we find we can endure the dark and difficult times. When the light returns, we cherish it and discover how to love truth more fully.














Live your life well in accord with the Way --


avoid a life of distraction.


A life well-lived leads to contentment,


both now and in the future.





V. 169





With a heart of contentment as our foundation, we can tackle the tasks that confront us. There are times we need to be brave warriors battling with the forces of delusion to avoid their taking control of our hearts and minds. At other times we need to be like a parent, nurturing and caring for the goodness that is already here. Agility is a great spiritual virtue. Recognising the beauty inherent in a contented heart, we will naturally be drawn towards it. We only seek distraction because we don't know contentment. Right practice liberates energy previously consumed by compulsiveness. This same energy can also manifest as vitality and enthusiasm. 109














Just as a sweet scented and beautiful lotus
can grow from a pile of discarded waste,
the radiance of a true disciple of the Buddha
outshines dark shadows cast by ignorance.





v.  58 - 59





We tend to see the things we don't like about ourselves and others as obstructions to our happiness. `How radiant, how fulfilled I would feel, how acceptable I would become if only I didn't have all these limitations!' But everything we experience can be turned around and can nourish growth towards that which is inherently, timelessly beautiful. All the unfair, unwanted, unreasonable stuff of our lives, all the bits that we resist and reject, make up the swampland of discarded waste. The beautiful lotuses take root and grow in the unbeautiful. 














It is wisdom


that enables letting go


of a lesser happiness


in pursuit of a happiness


which is greater.





v. 290





The filters of our preferences tragically limit our seeing. We want to let go of that which binds us but often our will fails us. Wise reflection can support will -- it is will's best friend. Will is not meant to do it alone. The verse encourages us to reflect on how letting go of our attachment to a lesser happiness can lead to gaining a greater happiness. Lost in our attachments, we see only that which we stand to lose by letting go. Wise thinking means we see what we stand to lose as well as what we stand to gain. Wise reflection opens and broadens our vision and enables pursuit of the goal.














Since ancient times it has been the case 


that those who speak too much are criticised, 


as are those who speak too little 


and those who don't speak at all. 


Everyone in this world is criticised. 


There never was, nor will there be,


nor is there now,


anybody who is only blamed or wholly praised.





v. 227-8





Whatever we do in this life, whatever we say (or don't say), we cannot avoid being criticised. The Buddha was blamed and criticised just like everyone else. To seek only praise and fear blame is fruitless. The only blame with which we really need to be concerned is that which is offered by the wise. If someone who lives impeccably criticises us, it is appropriate to heed what they say. But if we find that any criticism from anybody hurts us, then we need to look deeper. Reflect that in blaming, people are sending out the pain that they don't have the ability to contain; they express their hurt outwardly by finding fault with others. When we have the mindfulness and capacity to receive ourselves fully we are not inclined to blame anyone, neither ourselves nor others.














If you find a good companion,


of integrity and wisdom,


you will overcome all dangers


in joyous and caring company.





v. 328





Our mind, like water, takes on the shape of the vessel in which it is contained. The Teacher is encouraging us to be mindful of the company we keep. The Discourse on Great Blessings says, ``Avoid the company of the foolish and associate yourself with the wise''. As we apply skilful discrimination, we need to exercise care that we don't confuse prejudice with wise discernment. Wise discernment is compassionate and kind and is interested in protecting all beings from harm.














As a stormy wind cannot move a mountain of rock,


so one who contemplates the reality of the body,


who develops faith and energy,


is unmoved by Māra.





v. 8 





Māra is the manifestation of unenlightenment; the driving force of denial. Māra is the avoidance of reality and is expressed as compulsiveness, insensitivity, resentment. Our monastery in rural Northumberland is sometimes buffeted by winds that can be frightening but they are nothing compared to the threat posed by the forces of Māra. If we are to withstand the onslaught of our heedless habits, we must firmly establish our contemplation in an awareness of the body, in faith and in energy. It is the ability and willingness to come back, over and over again, to our body-based practice, remembering to live in ways that lead to selfless confidence that sustains our interest in discovering truth. This has the power to make us unshakable.














Self-mastery is the supreme victory --


much more to be valued than winning control over others.


It is a victory that no other being whatsoever


can distort or take away.





v. 104 - 5





Unshakably established in the centre of our lives, nothing can unsettle us or generate causes for suffering. This is fearlessness. The Buddha's teachings point to that which blocks fearlessness: habitual self-seeking, deluded desire. Shining the light of awareness on this very activity of desire -- not judging or interfering with it in any way -- we learn to see that each small moment of self-mastery leads towards the victory of selflessness.














Silence does not denote profundity


if you are ignorant and untrained.


Like one holding scales,


a sage weighs things up, 


wholesome and unwholesome, 


and comes to know both the inner and outer worlds. 


Therefore the sage is called wise.





v. 268 - 269





The Buddha spoke of the contentment and benefit that can come from living in quiet and beautiful places. Limiting sense stimuli can assist us on our path to freedom from ignorance. However he didn't mean for us to then take a position against the sensory world. Ajahn Chah often said: ``If you can't practice in the city, you can't practice in the forest''. And he'd also say: ``If you can't practice when you are sick, you can't practice when you are healthy''. In other words, everything is practice; including the feeling that we can't practice with `this'. It is wisdom that recognises this truth.














There are those who awaken from heedlessness. 


They bring light into the world


like the moon 


emerging from clouds.





v. 172 





On the full moon day of the month of Asalha, over two thousand five hundred years ago, the Buddha first revealed the Four Noble Truths. The hearts of those who heard these teachings were filled with the joy of perfect understanding, as cool and brilliant as a moon emerging from a covering of clouds. We assume the world and our suffering to be more substantial than it really is. But if we turn our attention towards the true causes of suffering -- desire rooted in ignorance - these Four Ennobling Truths generate a radiance that disperses the clouds of delusion and dissolves the world of suffering. Our contemplations awaken us from heedlessness, moment by moment we contribute to the continued turning of this Dhamma Wheel.














Do not seek the company of misguided friends;


beware of degenerate companions.


Seek and enjoy the company of well-guided friends, 


those who support insight. 





v. 78 





How do we hold our friendships and how do our friends hold us? We want to protect and nurture true friendship with those who support our aspiration to live in truth. We genuinely appreciate such companions; we don't wait until we get into difficulties before letting them know what they mean to us. Good friendship can be cultivated. As we bring mindfulness to our relationships with others let's also consider how friendly we are being with ourselves.














Those who build canals


channel the flow of water.


Arrowsmiths make arrows.


Woodworkers craft wood.


Those committed to goodness 


tame themselves.





v. 145 





Crafts-people work their crafts. The endeavour of taming the heart's unruly nature -- there's an art in that. Outwardly we may find ourselves in confusing circumstances but inwardly we remember that cultivating skilfulness is our primary task. With careful, constant observation we learn to recognise the heart's unruliness. `If only the weather was warmer!' `If only the weather was cooler!' We don't reactively judge what arises. Our passionate demanding that conditions be other than how they are is simply seen for what it is. There's an increased daring and willingness to offer ourselves into whatever it is we are doing right now; whole-heartedly, whole-bodily. As we progress, the task becomes easier; gratitude manifests, even though the work is challenging.














Why is there laughter?


Why is there joy 


when the world is on fire? 


Since you are clouded in darkness


should you not seek the light?





v. 146





It is usually when we get burned by life that we turn to the Teachings. It can come as a relief to find that in our efforts to be free from darkness we are accompanied by many millions of other human beings. Suffering is the nature of unawakened humanity. ``Don't feel bad if you are suffering. Everyone suffers,'' Ajahn Chah would tell us. Before his enlightenment the Buddha-to-be suffered too. The difference is enlightened beings know that suffering is not an obligation -- it is only one of the options available in the human realm. There is also the possibility of dwelling in the light of non-suffering.














Gradually, gradually,


a moment at a time,


the wise remove their own impurities 


as a goldsmith removes the dross.





v. 239





No amount of wishing things were otherwise gives us what we long for. We want the pure gold of pristine awareness so we need to enter the fires of purification. This verse instructs us on how to watch over the burning: too much heat -- we are trying too hard -- enduring heedlessly we get hurt in our practice. Not enough heat -- shying away from difficulties -- following preferences for comfort and ease there is no improvement in our practice. We just become more foolish as the years go by. Our habits are the dross and with gradual fine-tuning of our effort we learn letting go. The aim of all this work is the realisation of the state of luminous awareness. We then have something inherently valuable to share with others.














Better than ruling the whole world,


better than going to heaven,


better than lordship over the universe,


is an irreversible commitment to the Way.





v. 178





Unconditioned freedom: a quality of being not dependent on any condition whatsoever. Whatever the circumstances, fortunate or unfortunate, the heart remains at ease, radiant, clear-seeing, sensitive and strong. It is a commitment that is irreversible, unshakable and real; beyond all of deluded ego's obsessions. To arrive at this level of resolve requires constant observation of our old habits, for example: the fondness of being in control of everything, addiction to passing pleasures, obsession with power. We work with what we have. Each time we find ourselves distracted we refocus our commitment to the Way.














It is wisdom


that enables letting go


of a lesser happiness


in pursuit of a happiness


which is greater.





v. 229





We want to let go of that which binds us but often our will fails us. Wise reflection can support will -- it is will's dear friend. Will is not meant to do it alone. The verse encourages us to reflect upon how letting go of our attachment to a smaller happiness can lead to gaining a greater happiness. When we are lost in our attachments we only see that which we might lose by letting go. Wise thinking means we see what we stand to lose as well as what we stand to gain. The filters of our preferences tragically limit our seeing. Wise reflection opens and broadens our vision and enables pursuit of the goal.














All states of being are determined by mind.


It is mind that leads the way.


Just as the wheel of the oxcart follows 


the hoof-print of the animal that draws it, 


so suffering will surely follow 


when we speak or act impulsively


from an impure state of mind.





v. 1





Thinking too much can complicate spiritual practice; not thinking enough can restrict us. Here we are given an example of wisely-directed thinking. It helps us to see that we are not a victim of circumstance -- to understand that the intention behind our actions of body and speech determines our state of being -- it puts us in the position where we have the power to bring about real change. Accepting this teaching as our field of investigation we appreciate the value of exercising mindful restraint and find new confidence and ability. Our heart/mind leads the way.














Having performed a wholesome act


it is good to repeat it.


Enjoy the pleasure of its memory. 


The fruit of goodness is contentment.





v. 118





The Dhamma encourages us to bring to heart the memory of wholesome acts performed by body, speech or mind and to dwell in the sense of gladness that arises with that recollection. We are not at risk of becoming overly pleased with ourselves if mindfulness is present. We can similarly contemplate our mistakes and shortcomings without losing sight of our goodness if we have developed wise reflection. That we can observe our habits shows that we are much more than those habits. What is it that observes? This is our refuge - awareness - the way out of suffering. With this very awareness we can freely dwell in the joy of recollecting goodness. And what a relief to find that we can make mistakes and learn from them.














Let go of that which is in front,


let go of that which has already gone,


and let go of in-between.


With a heart that takes hold nowhere


you arrive at the place beyond all suffering.





v. 348 





We can comfortably lose ourselves in ideas of the future. We can comfortably dwell in memories of the past. And we readily get lost in experiences here and now. 


When we get lost we suffer. But suffering is not ultimate; there are causes for suffering and there is freedom from them. To arrive at freedom requires that we are willing to leave our familiar abidings. We hear the instruction to `let go' and it can sound like we are being told to get rid of something or that we are wrong for being the way we are. But if we recognise the power of right mindfulness, we see letting go just happens. In this verse the Buddha meets us in the place of our suffering and directs us towards an open door out of it.














The sun shines by day,


the moon shines by night.


But, both all day and all night


the Buddha shines 


in glorious splendour.





v. 387





Whenever there is well-established mindfulness, there is beauty, clarity and the possibility for deepening understanding. If there is constant mindfulness, there is constant clarity. All day and all night the heart of mindfulness shines in glorious splendour.














Do not ignore the effect of right action


saying, ``This will come to nothing.''


Just as by the gradual fall of raindrops


the water jar is filled,


so in time the wise become replete with good.





v. 122





 `Bigger is better'. `The more the merrier'. Little by little such perceptions can be wisely questioned. Trust that every small moment of mindfulness matters. Every small effort to remember to come back and with judgment-free awareness to begin again makes a difference. No such effort is ever wasted. One day we discover we are simply unimpressed by that which previously held us. Instead of reacting to something that in the past would upset us, letting go happens. Wisdom knows the way of true goodness.














If spoken to harshly,


make yourself as silent as a cracked gong;


non-retaliation is a sign of freedom.





v. 134


When we receive unjustified criticism it can be hard to restrain the up-thrusts of passion. Pushing strong feelings down into unawareness isn't helpful. Practice means finding the space within ourselves to feel what we feel, without `becoming' those feelings. It is a special skill. Watch out for any voice preaching at you: ``You shouldn't be this way, you should know better by now.'' We acknowledge the fact of how it is in this moment. Accept the present reality into awareness, knowing it as it is; not indulging, not pushing away. Now the energy of our passions can fuel the process of purification, burning out the pollutions rather than burning us out with self-criticism.














Having empathy for others


one sees that all beings


love life and fear death.


Knowing this,


one does not attack or cause attack.





v. 130





By being open to others we bring benefit to ourselves. In studying our own hearts and minds we generate benefit for others. Dhamma teaches us to practice on both levels. On one level we cultivate insight that undoes deeply held views about `self' and `others'. This is our formal practice. Meanwhile our daily-life practice calls us to accept the relative validity of views like `self' and `others'; to engage them in the cultivation of compassion and kindness. Empathy is a companion of insight and helps keep practice balanced.














Those who are envious, stingy and manipulative,


remain unappealing despite good looks and eloquent speech.


But those who have freed themselves from their faults 


and arrived at wisdom are attractive indeed.





v. 262-263





We need to remind ourselves that the way things appear to be is not necessarily the way things `are'. If we listen from the heart to someone as they speak, instead of listening only with our ears, we will hear something quite different. When we view others from a quiet centred place deep within, we see much more than that which our eyes see.














Committed evil doers behave toward themselves


like their own worst enemies.


They are like creepers that strangle the trees


which support them.





v. 162





This verse refers to a monk who tried three times to kill the Buddha and eventually his evil actions contributed to his own death. When we betray our heart's commitment to reality, slowly but surely, we die away from the light of truth and sink into darkness. Creepers climb handsome, mature trees and sometimes strangle them to death. We can go for refuge to the Buddha yet still be taken over by up-thrusts of rage. Hours, days or even years go by as we justify our hurtful actions of body or speech. When we come to see the truth of our actions a wholesome sense of remorse arises; we genuinely wish to desist. Right action is the natural consequence.














A renunciate does not oppress anyone.


Patient endurance is the ultimate asceticism.


Profound liberation, say the Buddhas, 


is the supreme goal.





v. 184 





As we progress on our path towards the goal, the Buddha is encouraging us not to forget how we relate with others. The cultivation of metta teaches us to include all beings in our heart of kindness. It sometimes happens that we feel able to be kind towards others but not towards ourselves. If we find this a struggle, we show infinite kindness towards the struggler, ourselves. Layers of self-judging, self-loathing, self-rejecting require us to exercise patient endurance; not to be mistaken with bitter endurance. Patience is an essential characteristic for those travelling on the path. And there is absolutely no way this essential virtue can be cultivated when we are having a good time. So when we are having a bad time, reflect that this is the perfect place, the only place, to develop this profound quality.














Those who are contentious 


have forgotten that we all die;


for the wise, who reflect on this fact, 


there are no quarrels.





v. 6 





It is OK to think about death and to talk about the fact that we all will die. Indeed it is wise to do so. If we deny the awareness of our mortality, the fear of it remains hidden underground. With this avoidance of reality comes diminishing aliveness and increased confusion. Cultivating mindfulness around the subject of death (maranasati) we discover that to the degree we admit reality there is a corresponding increase in clarity and contentment. This is the contentment that leads to meaningful peace.





\texttt{}

\texttt{}

\texttt{}

All states of being are determined by mind.

It is mind that leads the way.

As surely as our shadow never leaves us,

so well-being will follow when we speak or act

with a pure state of mind.


v. 2





Have you ever tried to run away from your shadow? As hard as you might try it never leaves you. Any action by body, speech or mind that is done with pure motivation, will indeed lead to increased well-being. This is worth remembering if we intend to do something good and are about to change our minds, thinking it's not worth the effort. However small and apparently insignificant the act might be, happiness will follow. It is worth doing.














When we hold fast to such thoughts as,


``They abused me, mistreated me, 


molested me, robbed me,''


we keep hatred alive.





v. 3 - 4





If we thoroughly release ourselves 


from such thoughts as,


``They abused me, mistreated me, 


molested me, robbed me,''


hatred is vanquished.





In various ways we all suffer injustice in our lives. Sometimes the pain runs deep and can last for years. Dhamma teachings emphasise not so much the pain but our relationship to it. As long as we are possessed by hatred and resistance, intelligence is compromised. Although action might be called for, if our heart is not free from hatred, we can't know what right action would be. The Buddha advocates releasing ourselves from thoughts of hatred. It takes strength, patience and determination to let go. We are not letting go because someone else told us we should. We are letting go because we understand the consequences of hanging on.














As old flowers fall 


from a jasmine plant


let lust and hatred


fall away.





v. 377





Ajahn Chah was in London, staying at the Hampstead Vihara. The monks were troubled by the noise that was coming from the pub across the road. Ajahn Chah told them that the cause of suffering was their sending attention out to trouble the sound. Sound itself is just so. Suffering only arises when we `go out' and add something extra. Seeing our part in creating problems, a shift in the way we view struggles takes place. Instead of blaming we simply `see' what we are doing, in the moment. Let's not get into a fight with hatred; exercising careful restraint and wise reflection, we let it `fall away'. Initially we see this only after we have reacted and created suffering. With practice we catch it sooner. One day, we will catch ourselves just as we are about to create the problem.














As water slides from a lotus leaf,


so sensual pleasures


do not cling


to a great being.





v. 401





A great being is great because he or she is free from obstructions in their relationship with life. We are not so great because we get caught in feelings and make a problem out of life. We create obstructions by the way that we deal with the eight worldly dhammas: praise and blame, gain and loss, pleasure and suffering, popularity and insignificance. Out of delusion we relate to these worldly winds heedlessly - indulging in what we like and resisting what we don't. Wisdom on the other hand simply sees the reality of the sensory world. It knows the space within which all experiences arise and cease. Such knowing means a great being doesn't even have to try to let go; all inclination to cling automatically falls away. He or she experiences sensual pleasure but adds nothing to it and takes nothing away.














Never by hatred is hatred conquered,


but by readiness to love alone.


This is eternal law.





v. 5





To conquer hatred `by love alone' can seem a lofty ideal. We might judge ourselves (and in so doing tend to excuse ourselves) as being too limited -- or just dismiss the notion as not realistic. Surely part of us does say, `yes'. Another part of us might say, `Yes, but what if\ldots{}' For the Buddha there are no qualifications; hatred simply never works. What is this word `love'? Generating an unqualified receptivity? And the perceptions we have of those who might trigger hatred within us; where do those perceptions exist? Potentially loving-kindness is capable of receiving all of this into awareness, fully, wholly, even warmly. This is eternal law.














Flamboyant outer appearance


does not in itself constitute an obstruction to freedom.


Having a heart at peace, pure, contained, 


awake and blameless,


distinguishes a noble being.





v. 142 





Outer forms are not what really matter. The Buddha always pointed to the heart, the place we need to focus. He emphasized this because we easily forget, becoming too concerned with how things appear. This verse is about an inebriated householder shocked by massive disappointment into despair. The Buddha's teachings, pointing directly at that which really matters, transformed his despair; instead of falling apart, he realised perfect unconditioned peace. When the focus of our practice is merely on forms and appearances, spirit is compromised. Straining too much for the ultimate explanation of the five precepts, for example, may stop us from seeing the intention behind the action. The form of the precepts is there so we become mindful of our motivation. Holding the precepts correctly, there is a chance they will serve their true purpose. Correct living is noble being.














There are those who discover 


they can leave behind confused reactions 


and become patient as the earth;


unmoved by anger,


unshaken as a pillar,


unperturbed as a clear and quiet pool.





v. 95





The Buddha lived in this world as we do. And despite all the turmoil he realised a state of imperturbability - to `become patient as the earth'. Whatever is poured on it, burnt on it, or done to it, the earth just goes on being what it is, doing what it does. Patient endurance is not a weak option; it is strengthening and kind. With patient endurance we discover the ability to allow our present experience to be here, just as it is, until we have learned what we need to learn.














Making an arbitrary decision


does not amount to justice.


Having considered arguments 


for and against, 


the wise decide the case.





v. 256





You are under pressure to make a decision. Is it possible to remain cool and calm when others want you to decide in their favour? Can you remain free from bias and arrive at a just decision? How do we hold our views? Having a strong opinion can feel great; it can appear as confidence. But such is the nature of fundamentalism; so too is providing simplistic answers to complex questions. Rigid views and simplistic solutions are not aspects of a spacious mind; a mind that can consider all aspects of a dilemma. It usually takes time to arrive at a balanced, thoroughly considered view. It also requires an ability to listen from a place of inner quiet. If we are mentally preparing our rebuttal we are not really listening.














Even those who live wholesome lives 


can experience suffering so long as their acts 


have not yet borne direct fruits. 


However, when the fruits of their actions ripen


the joyful consequences cannot be avoided.





v. 120





Trusting in true principles is not always easy. Pursuing popular opinion is not always wise. Not being dissuaded by doubts or mere speculation, we see how our deepest questions about reality lead us to insight. Trust is a force for transformation, so too is patience. It is almost unimaginable how an acorn can grow into an oak tree, but it does. We need trust. We need patience. Like trees, these forces for transformation can be cultivated and inspire us towards that which is inwardly great and inherently free of all suffering.














Alert to the needs of the journey,


those on the path of awareness,


like swans, glide on,


leaving behind their former resting places.





v. 91





It is time to move on. We might know this on one level yet another part of us holds back. If we attempt to let go because we think we should -- instead of discovering renewed inspiration -- there will be resistance. But encountering disappointment is OK too if we are alert to our heart's true needs. Our deepest need is to be free from ignorance. There is no place we can truly settle until we have fully realised who and what we are. This path of awareness has the power to transform absolutely everything into understanding. Swans glide on -- no resistance.














Transform anger with kindness 


and evil with good,


meanness with generosity


and deceit with integrity.





v. 223





If we are cold, we find a way to get warm -- we don't subject ourselves to more cold. If we are hungry we eat -- we don't further deprive ourselves of food. If we are angry, we don't fight it with rage -- we try to be kind to this being that is suffering from anger. If we witness evil actions, we generate sincere goodness and restrain any impulse to reject the evil-doers with mere judgment. To people who believe self-centred meanness is the path to contentment we give generosity. And with those who are deceitful we speak truth. This may not be easy. This is the way of transformation.














\texttt{Those who speak much}

\texttt{are not necessarily possessed of wisdom.}

\texttt{The wise can be seen to be at peace with life}

\texttt{and free from all enmity and fear.}

\texttt{}

\texttt{258}




Struggling to become peaceful and wise often stirs things up while carefully allowing the activity of heart and mind to cease naturally calms things down. Wisdom is not discovered by following habits of seeking that which we like. When pain and disappointment come to us we can learn to meet them with a quality of willingness and awareness that leads to understanding. When happiness and delight come we can be refreshed and renewed by our experience without becoming lost in them. Enmity and fear are not necessarily obstacles. They can be our teachers pointing to a more spacious and wise way of living.











On hearing true teachings,


the hearts of those who are receptive


become serene, like a lake, 


deep, clear and still. 





v. 82 





We might think, `If the Buddha were here today to teach me, I too could get enlightened.' However when the Buddha was around there were many who met him, heard him, lived with him and yet didn't recognise him or the Truth to which he was pointing. At the end of his life one of the monks asked the Buddha who would take his place when he was gone, to which he replied that the Dhamma would take his place. On an earlier occasion he had taught that seeing the Dhamma was the same as seeing the Buddha. Listening to Dhamma is attending fully to and contemplating wisely that which is happening right now. So we don't need to feel bad because we can't listen to a talk directly from the Buddha. We learn to be more receptive to life.














One who transforms old and heedless ways


into fresh and wholesome acts


brings light into the world


like the moon freed from clouds.





v. 173 





The darkness of the world comes from habits of avoidance and indulgence. Wholesome acts uncover the light that is already here. Paying attention to old worn-out ways in the right manner at the right time with here-and-now, judgment-free, body-mind awareness leads to transformation. We don't have to get rid of anything. We don't have any bad stuff. Painful memories, painful sensations do not have to become suffering. With right attention everything leads to increased light.














Not in great wealth is there contentment,


nor in sensual pleasure,


gross or refined.


But in the extinction of craving


is joy to be found by a disciple of the Buddha.





v. 186--187 


 


Is what I am doing with my life giving me what I am looking for? We do succeed from time to time in gratifying our wishes but lasting happiness comes only from being freed of the irritation of wanting. If we ask ourselves the right questions at the right time, a radically different view is revealed: the way of contentment draws us inwards, not outwards. Rather than following an impulse to want more, we can look directly at the wanting itself. And for a moment the sting of craving ceases. We have learned a little more about what it means to be a disciple of the Buddha.














As rain cannot penetrate


a well-thatched roof,


so the passions


cannot enter a well-trained heart.





v. 14





We act to protect ourselves from the elements of wind and rain. With wise reflection we act to protect ourselves from the wild passions. This is heart-training. Skilfully addressing the damage caused by greed and ill will is an important aspect of Right Action. It is not an avoidance of the passions; neither is it indulging in them. Between these two we seek the middle way.














A well-trained horse


gives no cause for restraint.


Rare are those beings who,


through modesty and discipline,


give no cause for rebuke.





v. 143


 


Is it possible to have too much modesty? It is possible to have the wrong kind of modesty; when, for example, refraining from blatant greed is being manipulative. Modesty, frugality and discipline: words like these can make us feel uneasy. Yet certainly there are timeless principles encoded in them. When we have true modesty we look for the `right amount' of things. We seek the difference between settling for `good enough' and being too timid to excel. With right discipline we are focussed on the task at hand without compromising sensitivity. One skilled in right discipline can say no when needed, not out of judgement or mere preference, but as someone who cares.














As a bee gathering nectar 


does not harm or disturb 


the colour and fragrance of the flower,


so do the wise move


through the world.





v. 49





The world is our village. Can we move through our world without disturbing that which is beautiful about it? Sages and renunciates might live on alms-food gathered from the village but most people need to use money. There is nothing inherently wrong about money. It is a symbol for the energy with which we interact in our world. The Buddha's guidelines on cultivating Right Livelihood indicate harmless and considerate ways of stewarding this energy. When we forget ourselves we focus too much on how good it will be when we get what we want, not attending to the way we go about getting it. Care is needed both in what we do and how we do it.














Naturally loved are those 


who live with right action 


and have found the Way, 


and through insight 


have become established in the truth. 





V. 217





When actions of body and speech are an expression of inner ease, we find the realm of relationships more rewarding. It's not by merely following our wish to be liked or any other gratification of desire that we live with lasting happiness; rather it is by appreciating where the true causes for contentment lie. Even wholesome wishes if held too tightly can lead to discontentment. It's clinging that's the problem. When we see clearly our habits of clinging, at that time and in that place we also see how we don't have to cling; we find the way out.














Knowing only a little about Dhamma


but wholeheartedly according with it,


transforming the passions of greed, hatred and delusion,


releasing all attachments to here and hereafter,


one will indeed experience for oneself the benefits of walking the Way.





v. 20





It doesn't matter that we don't know all there is to know about Dhamma. It is how we apply the teachings that counts. Are we living wholeheartedly according with what we have learned? This is a more useful line of enquiry than, `When can I go on another retreat?' or `If I could only knuckle down to more study?' Such questions are OK if they lead us to releasing attachments but not if they are an expression of our addiction to always wanting or becoming more. If what is needed is a subtle shift of attention to simply being more mindful in the present moment, even striving hard for profound insights obstructs our progress. A moment of recognising this truth and we experience for ourselves the benefit of walking the Way.














Beings naturally experience pleasure;


but when pleasure is contaminated with craving,


not releasing it creates frustration 


and tedious suffering follows.





v. 341





With the right quality of mindfulness at the right moment, we can see why and how attaching to joy and sorrow perpetuates suffering. We become interested in how to experience pleasure without making it `my' pleasure or creating an `I' that is having a good time. Wise reflection shows us that when attachment is taken out of the experience happiness isn't diminished -- in fact it is enhanced. When we are accurately aware, intelligence functions in the service of insight. If we appreciate how grasping at the pleasure spoils the beauty, an ease of being emerges.














To many places beings withdraw 


to escape from fear:


to mountains, forests, parklands and gardens; 


sacred places as well.


But none of these places offer true refuge,


none of them can free us from fear.





v. 188 - 189 





It is hard to feel afraid without thinking something is going wrong. We readily react by judging ourselves and others, in an attempt to escape the pain of fear. It doesn't work -- neither does running off into the wilds. Even sacred places are deemed to fail us if we are motivated by a wish to escape. Turning to our refuge in Dhamma however can trigger an interest in understanding fear and learning from fear. Can we experience the fear sensation without `becoming' afraid? Fear is still fear but it is perceived from an expanded, less cramped and less threatened awareness. We can even begin to see that fear too is `just so'. A non-judgemental, whole body-mind acknowledgement of the condition of fear, here and now, can transform our pain into freedom. Willingness to meet ourselves where we find ourselves is the way.














It is hard to find 


a being of great wisdom;


rare are the places in which they are born.


Those who surround them when they appear


know good fortune indeed.





v. 193





Associating with wise beings is certainly a blessing but it is not always easy. Their openness can contrast painfully with our contractedness. However, wise beings are also compassionate so it is easier to take the medicine. Eventually we feel grateful. After his enlightenment the Buddha felt so much gratitude for his previous teachers that one of the first things he did was try to find them and pass on what he had discovered. We too can reflect with gratitude on all those who have helped us take steps on this path. Dwelling on gratitude it is heart nourishment; it aligns us with the wise.














The Buddha's perfection is complete;


there is no more work to be done.


No measure is there for his wisdom;


no limits are there to be found.


In what way could he be distracted from truth?





v. 179





There is a real reality that can be known; the Buddha knew it. It is possible to awaken beyond all the feelings of limitation that lead to disappointment and regret. Skilfully trusting in the Buddha's enlightenment establishes an inner frame of reference. When we find ourselves disoriented by life we realign ourselves by engaging this trust. Trust is an inner structure, like the sails on a yacht; when rightly set they capture the wind which powers the vessel onwards.














For many lives I have wandered


looking for, but not finding,


the house-builder who caused my suffering.


But now you are seen and you shall build no more.


Your rafters are dislodged and the ridge-pole is broken.


All craving is ended; my heart is as one with the unmade.





v. 153-4





The Buddha's final and complete realisation was to discover that he had been believing in something that was not true. He had been deceived by what he calls the house-builder. The houses are the structures of mind; the `me' and the `mine' we take so seriously. `It is me that is longing.' `It is me that is feeling disappointed.'  `It is my mood, my body, my mind.'  He saw clearly how these impressions had been constructed by the habits of craving. Having penetrated this with profound insight, the process was understood; the mainstay of the house, the ridge-pole, was broken and suffering had ended. Wandering around hoping for a solution to his struggles was over. From here on, he would abide at ease in the state prior to all arising and ceasing; the unmade, the undying reality.








\_\_\_\_\_\_\_\_\_\_\_\_\_\_\_\_\_\_\_\_\_\_\_\_\_\_\_\_\_\_\_\_\_\_\_\_\_\_\_\_\_\_\_\_\_\_\_\_\_\_\_\_\_\_\_\_\_\_\_\_\_\_\_\_\_\_\_





Index of first lines.








v. 1 All states of being are determined by mind ... p.


v. 2 All states of being are determined by mind ... p.


v. 3 - 4 When we hold fast to such thoughts as ... p.


v. 5 Never by hatred is hatred conquered ... p.


v. 6 Those who are contentious ... p.


v. 8 As a stormy wind cannot move a mountain of rock ... p. 


v. 14 As rain cannot penetrate ... p.


v. 20 Knowing only a little about Dhamma ... p.


v. 49 As a bee gathering nectar ... p.


v. 58 - 59 Just as a sweet scented and beautiful lotus ... p.


v. 78 Do not seek the company of misguided friends ... p.


v. 82 On hearing true teachings ... p.


v. 87-88 With an image of liberation as the goal ... p.


v. 91 Alert to the needs of the journey ... p.


v. 95 There are those who discover ... p.


v. 104 - 5 Self-mastery is the supreme victory ... p.


v. 118 Having performed a wholesome act ... p.


v. 120 Even those who live wholesome lives ... p.


v. 122 Do not ignore the effect of right action ...p.


v. 130 Having empathy for others ... p.


v. 134 If spoken to harshly ... p.


v. 142 Flamboyant outer appearance ... p. 


v. 143 A well-trained horse ... p.


v. 145 Those who build canals ... p.


v. 146 Why is there laughter ... p.


v. 153-4 For many lives I have wandered ... p.


v. 162 Committed evil doers behave toward themselves ... p.


v. 169 Live your life well in accord with the Way ... p.


v. 172 There are those who awaken from heedlessness ... p.


v. 173 One who transforms old and heedless ways ... p.


v. 178 Better than ruling the whole world ... p.


v. 179 The Buddha's perfection is complete ... p.


v. 184 A renunciate does not oppress anyone ... p.


v. 186--187 Not in great wealth is there contentment ... p.


v. 188 - 189 To many places beings withdraw ... p.


v. 193 It is hard to find a being of great wisdom ... p.


v. 217 Naturally loved are those ... p. 


v. 223 Transform anger with kindness ... p.


v. 227-8 Since ancient times it has been the case ... p.


v. 229 It is wisdom that enables letting go ... p.


v. 239 Gradually, gradually, a moment at a time ... p.


v. 256 Making an arbitrary decision ... p.


v. 258 Those who speak much ... p.


v. 262-263 Those who are envious ... p.


v. 268 - 269 Silence does not denote profundity ... p.


v. 290 It is wisdom that enables letting go ... p.


v. 328 If you find a good companion ... p.


v. 341 Beings naturally experience pleasure ... p.


v. 348 Let go of that which is in front ... p.


v. 377 As old flowers fall ... p.


v. 387 The sun shines by day ... p.


v. 401 As water slides from a lotus leaf ... p.





\_\_\_\_\_\_\_\_\_\_\_\_\_\_\_\_\_\_\_\_\_\_\_\_\_\_\_\_\_\_\_\_\_\_\_\_\_\_\_\_\_\_\_\_\_\_\_\_\_\_\_\_\_\_\_\_\_\_\_\_\_\_\_\_\_\_\_











``The Gift of Dhamma Excels All Other Gifts ''





This book is offered for free distribution in the


trust that it will be of benefit to those interested in


entering into a deeper contemplation and appreciation


of their lives. We at Aruna Publications feel


privileged to be able to produce books such as


this and look forward to continuing to do so. We


would be delighted to receive any comments or


feedback from readers of this volume.





The cost of this publication has been met


entirely from donations, thereby providing 


the much-appreciated opportunity


for sponsorship and dedications of offering in


accordance with Buddhist tradition. 


Donations made towards Dhamma publications 


will contribute towards similar books in the future. 


If you value the chance to be involved 


in sponsoring ``Gifts of Dhamma''


please do not hesitate to contact:





Aruna Publications,


Aruna Ratanagiri Buddhist Monastery,


2 Harnham Hall Cottages,


Harnham, Belsay,


Northumberland


NE20 0HF


UK





Email: \href{mailto:aruna.publications@ratanagiri.org.uk}{aruna.publications@ratanagiri.org.uk}





\_\_\_\_\_\_\_\_\_\_\_\_\_\_\_\_\_\_\_\_\_\_\_\_\_\_\_\_\_\_\_\_\_\_\_\_\_\_\_\_\_\_\_\_\_\_\_\_\_\_\_\_\_\_\_\_\_\_\_\_\_\_\_\_\_\_\_





Copyright details











\_\_\_\_\_\_\_\_\_\_\_\_\_\_\_\_\_\_\_\_\_\_\_\_\_\_\_\_\_\_\_\_\_\_\_\_\_\_\_\_\_\_\_\_\_\_\_\_\_\_\_\_\_\_\_\_\_\_\_\_\_\_\_\_\_\_\_





Notes











\_\_\_\_\_\_\_\_\_\_\_\_\_\_\_\_\_\_\_\_\_\_\_\_\_\_\_\_\_\_\_\_\_\_\_\_\_\_\_\_\_\_\_\_\_\_\_\_\_\_\_\_\_\_\_\_\_\_\_\_\_\_\_\_\_\_\_





Notes











\_\_\_\_\_\_\_\_\_\_\_\_\_\_\_\_\_\_\_\_\_\_\_\_\_\_\_\_\_\_\_\_\_\_\_\_\_\_\_\_\_\_\_\_\_\_\_\_\_\_\_\_\_\_\_\_\_\_\_\_\_\_\_\_\_\_\_





Notes








\end{document}
