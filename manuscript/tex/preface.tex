
\chapter{Preface}

It has long been the tradition in Buddhist countries for the laity to
come to the local monastery each new-moon and full-moon to hear a Dhamma
talk. Indeed, the Buddha himself encouraged his Sangha to maintain this
fortnightly practice. When it was suggested to me that I could use the
internet to send out tracts of Dhamma each new-moon and full-moon, I
wasn't sure about the idea, however, I decided to give it a go. Although
we live in a world where the phases of the moon no longer mean much, for
many it still helps to be reminded of the ancient tradition of which we
are a part.

In September 2007 we started to send out verses from the Dhammapada
selected from `A Dhammapada for Contemplation', 2006. Each `moon day'
one verse was offered, supported by a short reflection on the verse.
This programme is now quite well known, passed on from one person to the
next by word of mouth and Cc-ed emails. I hear from people in different
parts of the world who appreciate receiving a timely reminder of the
ancient way as they go about their busy lives. Others look forward each
new and full-moon evening to opening their emails when they get home
from work. They are used privately, copied extensively, translated, and
passed around. I hear they also form the basis for discussion at some
weekly meditation group meetings.

It has been my intention that by sharing my personal reflections in this
way, others might feel encouraged to engage their own contemplative
ability. There is a tendency perhaps for Buddhist practitioners in the
West to try to find peace and understanding by stopping all thinking.
Yet the Buddha tells us that it is by \emph{yoniso manasikāra}, or wise
reflection, that we come to see the true nature of our minds; not
through just stopping thinking.

I am indebted to many who have helped in the preparation of this
material. For the Dhammapada verses themselves I consulted several
authoritative versions. In particular I have used the work by Ven.
Narada Thera (B.M.S. 1978), Ven. Ananda Maitreya Thera (Lotsawa 1988),
Daw Mya Tin and the editors of the Burmese Pitaka Association (1987) and
Ajahn Thanissaro. For the recorded stories associated with the verses I
also turned to \href{http://www.tipitaka.net/}{www.tipitaka.net}.

When I had heard from enough people that a book version of these
reflections would be useful, it was to my good friend Ron Lumsden I
turned. His considerable editing skill has helped craft my work, making
it ready for a more wider audience.

May the blessings that arise from the compilation of this small book be
shared with all who have been involved in its production and
sponsorship.  May all who seek the way find it and experience the
freedom at its end. May all beings seek the way.

\bigskip

{\par\raggedleft
Bhikkhu Munindo\\
Aruna Ratanagiri Buddhist Monastery\\
Northumberland, UK\\
Rainy Season Retreat (\emph{vassa}) 2009
\par}


