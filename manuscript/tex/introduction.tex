
\chapter{Introduction}

The radiance of the Buddha's wisdom reaches us here today, lighting the
path that guides us towards freedom from suffering. The luminosity of
the Dhamma is reflected in the words and deeds of those who follow the
Buddha's Teachings. Millions of men and women have caught a glimpse of
this light in the midst of our dark world. Guided by the Teacher's words
they have come to recognise the warm-heartedness and clarity that is
their true nature.

This book contains a selection of 52 verses taken from The Dhammapada.
Accompanying each verse is a short paragraph by Ajahn Munindo, a
Theravadin Buddhist monk and currently abbot of Aruna Ratanagiri
monastery in Northumberland, UK. The complete Dhammapada contains 423
verses, each a timeless example of the Buddha's radiant wisdom. They
are: 'very old artefacts that miraculously contain within them the
profound utterances of the Buddha', as Thomas Jones says in the
Afterword of ``A Dhammapada for Contemplation'' 2006, the publication
from which these verses are taken.

The message of both the original verses and the comments Ajahn Munindo
offers are received in a form that is neither preaching nor doctrinal.
The verses are translations from an oriental language that is by its
nature, culturally indirect. The format of the verses is curious and
interesting, having the quality of the Japanese haiku or stanzas from
the I Ching. They are inductive; they `induce' understanding in the
reader. At times the sequence of the lines in a verse appears inverted.
There's a sense that their beginnings should be at the end because the
examples come before the underlying thought of the verse is stated; as
we can see in verse 377: ``As old flowers fall from a jasmine plant, let
lust and hatred fall away.'' The inductive style is softer than English
readers are used to; here we are more gently led towards the main
thought of the verse: ``\ldots{} let lust and hatred fall away.'' Western
languages in general use the deductive style, where the reader is more
firmly directed to understanding. We can get an idea of this in verse
377, if we take the theme of the verse and place it at the beginning:
``Let lust and hatred fall away like old flowers falling from a jasmine
plant.''


The Dhammapada verses are aesthetic and indirectly instructive: ``Let go
of that which is in front, let go of that which has already gone, and
let go of in-between. With a heart that takes hold nowhere, you arrive
at the place beyond all suffering.'' v. 348. The point of arrival (the
place beyond all suffering) comes at the end of the verse. This
inductive structure naturally enables the reader to reflect on `the
arrival', having let go of everything.

Ajahn Munindo's comments are presented in this same inductive form; they
invite participation. Spiritual instructors of all persuasions can tend
to talk too much about the possibility of the journey rather than
joining us as we travel on the way. Readers of this book have an
opportunity to see how being part of a shared enquiry can naturally give
rise to new understanding; the process introduces us to a new part of
ourselves. The Buddha made it very clear that it was up to us to make
the effort: ``I can but point the way''. Dhammapada Reflections helps us
to make this effort. In his preface Ajahn Munindo identifies wise
reflection (yoniso manasikara) as a primary element of the spiritual
journey. His comments plant the seeds for reflection. We are left to
nurture and see them grow.

Perhaps you will decide to leave this volume open on your shrine table
with a new verse to ponder on each week or maybe carry it with you as
you travel. As one who has been studying these verses for many months
now, I am sure you will come to find they are inspirational. They are
beautiful things; their gently instructional nature simply and
skillfully directs us. Step by step we approach Dhamma: that which has
the power to enhance all aspects of our lives.

On behalf of all the recipients of the fortnightly Dhammasakaccha
messages, I would like to express our gratitude to Ajahn Munindo for the
time and energy he has put into the preparation of the material for this
publication. That which was previously available by email to a few
hundred individuals can now be read and passed around in the form of
this beautiful book.

{\par\raggedleft
Ron Lumsden, Little Oakley, Essex, 2009
}

